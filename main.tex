\documentclass{beamer}
%
% Choose how your presentation looks.
%
% For more themes, color themes and font themes, see:
% http://deic.uab.es/~iblanes/beamer_gallery/index_by_theme.html
%
\mode<presentation>
{
  \usetheme{default}      % or try Darmstadt, Madrid, Warsaw, ...
  \usecolortheme{default} % or try albatross, beaver, crane, ...
  \usefonttheme{default}  % or try serif, structurebold, ...
  \setbeamertemplate{navigation symbols}{}
  \setbeamertemplate{caption}[numbered]
} 


\usepackage[english,russian]{babel}
\usepackage[utf8x]{inputenc}

\title[Your Short Title]{Об оценках обобщенной размерности Хаусдорфа.}
\institute{ }
\author{Брыгин Сергей 621 гр.}
\date{ }

\begin{document}

\begin{frame}
  \titlepage
\end{frame}

\section{Доклад}


% 
%
%
\begin{frame}{Предварительные определения и утверждения}
    $(X, \Omega, h)$ - размерностное пространство, если:
    \begin{enumerate}
        \item  X - бесконечное множество
        \item $\varnothing \in \Omega$ - семейство подмножеств $X$, являющееся его покрытием
        \item $h : \Omega \to [0, \infty]$; $h(\varnothing) = 0$
    \end{enumerate}
\end{frame}

%  Первый функционал мы будем называть функционалом Хаусдорфа пространства $(X, \Omega, h)$;
% он является внешней мерой на $X$ 
% 
% Второй же функционал был определен в несколько меньшей общности в статье одного из авторов.
% Без дополнительных предположений о размерностном пространстве он не является внешней мерой;
% поэтому мы будем называть его функционалом Хаусдорфа-Лебега.
\begin{frame}{Предварительные определения и утверждения}
    Рассмотрим два функционала $\mu_H (E)$ и $\mu_{LH} (E)$ вида:
    $$\mu(E) = \inf{\sum{h(\Delta_i)}}$$
    где
    \begin{enumerate}
        \item[$\bullet$] $\mu_H$ : $\inf$ распространяется на все счетные покрытия множества $E$
        \item[$\bullet$] $\mu_{LH}$ : $\inf$ распространяется на все счетные и дизъюнктные покрытия множества $E$
    \end{enumerate}
\end{frame}

% Простейшим случаем совпадения двух функционалов при любой функции $h$ является случай, 
% когда выполнено следующее условие:
% каждое счетное семейство элементов  $\Omega$ содержит дизъюнктное подсемейство с  тем же объединением.
%
% Этим свойством обладает, в частности, семейство $\Omega_*$ всех  диадических кубов в $\mathbb{R}^n$, 
% где под диадическим кубом понимается любое произведение из $n$ 
% полуоткрытых справа интервалов (ребер) с началом каждого из них в одной из точек вида $\frac{l_i}{2^k}$ ,
% а с концом, соответственно, в точке $\frac{l_i + 1}{2^k}$
\begin{frame}{Предварительные определения и утверждения}
      $\Omega_*$ - семейство всех  диадических кубов в $\mathbb{R}^n$.
      
      Диадический куб есть следующее произведение
      $\overset{n}{ \underset{i = 1}{ \prod } } [ \frac{l_i}{2^k}; \frac{l_i + 1}{2^k} )$ 
      
      где $\{l_i\}_{i=1}^n$ - произвольные целые числа
      
      $k \in \mathbb{N}$ - ранг куба
\end{frame}

% В работе показано, что интересующие нас свойства функционала Хаусдорфа-Лебега справедливы 
% при фиксированных $X$ и $\Omega$ для произвольной функции $h$
% тогда и только тогда, когда  выполнено указанное выше условие
\begin{frame}{Предварительные определения и утверждения}
      \textbf{Теорема 1.} Следующие условия эквивалентны:
      \begin{enumerate}
        \item Для всякой функции $h: \Omega \to [0, \infty]$ со свойством 
              $h(\varnothing) = 0$ функционал $\mu_{LH}$ является внешней мерой. 
        \item Функционалы $\mu_H$ и $\mu_{LH}$ совпадают  при любой функции 
              $h: \Omega \to [0, \infty]$ со свойством $h(\varnothing) = 0$.
        \item Выполнены следующие два утверждения:
              \begin{enumerate}
                    \item Семейство $\Omega$ диадически упорядочено отношением включения, 
                                     то есть для любых $A, B \in \Omega$ , выполнено $A \subset B $
                                     или $ A \supset B$ или $A \bigcap B = \varnothing$.
                    \item Семейство $\Omega$ удовлетворяет условию обрыва возрастающих цепей, 
                                     то есть каждая возрастающая по включению последовательность элементов
                                     семейства $\Omega$ конечна.
              \end{enumerate}
        \item $\Omega$ удовлетворяет условию про существование дизъюнктного подсемейства.
    \end{enumerate}
\end{frame}

% 
% 
% 
\begin{frame}{Предварительные определения и утверждения}
    \textbf{Доказательство.} $4 \Rightarrow 2 \Rightarrow 1$ ясно. Покажем $1 \Rightarrow 3$.
    \newline
      
    Пусть 3.1 не верно: $\exists A, B \in \Omega: A \not \subset B, B \not \subset A, 
    A \bigcap B \ne \varnothing$. Положим $h(\varnothing) = 0, h(A) = h(B) = 1$
    и $h(E) = 3$ $\forall E \in \Omega$ . 
      
    Тогда $\mu_{LH} (A) = \mu_{LH} (B) = 1, \mu_{LH} (A \bigcup B) \geq 3 $ 
    и, значит, $\mu_{LH}$ -  не внешняя мера.
    \newline
      
    Пусть 3.2 не верно: $\exists \{A_n\}_{n=1}^{\infty} : A_n \subset A_{n+1}$. 
    Положим $h(\varnothing) = 0, h(A_n) = \frac{1}{n}, h(E) = 3$
    $\forall E \in \Omega$ .
    
    Тогда $\mu_{LH} (A_n) = 0;
    \mu_{LH} (\overset{\infty}{\underset{n=1}{\bigcup}} A_n ) \geq 3 $,
    $\mu_{LH}$ - не внешняя мера.
\end{frame}

% 
% 
% 
\begin{frame}{Предварительные определения и утверждения}
    \textbf{Доказательство.} Покажем $3 \Rightarrow 4$.
    \newline
      
    Рассмотрим $\{\Delta_i\}$ - произвольное счетное подсемейство элементов $\Omega$.
    
    По 3.2 можем взять для каждого $\Delta_i$ максимальный элемент $G_i$,
    которые либо совпадают, либо дизъюнктны.
    
    $\bigcup \Delta_i = \bigcup G_i$ чтд.
\end{frame}

%%%%%%%%%%%%%%%%%%%%%%%%%%%%%%%%%%%%%%%%%%%%%%%%%%%%%%%%%%%%%%%%%%%%%%%%%%%%%%%%%%%%%%%%%%%%%%%%%%
%% Основные определения %%
%%%%%%%%%%%%%%%%%%%%%%%%%%%%%%%%%%%%%%%%%%%%%%%%%%%%%%%%%%%%%%%%%%%%%%%%%%%%%%%%%%%%%%%%%%%%%%%%%%

%  Перейдем к основным определениям работы. 
% Заметим, что для любого размерностного пространства $(X,\Omega,h)$,
% удовлетворяющего эквивалентным условиям теоремы 1, 
% совокупность всех непустых элементов семейства $\Omega$  
%разбиватся на непересекающиеся классы следующим естественным образои. 
% Класс $\Omega_1$ состоит из всех максимальных (по включению) 
% элементов семейства $\Omega$, 
% класс $\Omega_2$ - из всех максимальных элементов семейства 
% $\Omega {\setminus}{ \Omega_1}$ и т.д. 
% Элементы класса $\Omega_k$ мы будем также называть в дальнейшем 
% элементами $\Omega$ ранга $k$ и писать $rg(\Delta)=k$ при 
% $\Delta\in\Omega_k$. 
%
% В общем случае ранг элемента является порядковым числом (ординалом), 
% но далее мы будем   рассматривать лишь однородные размерностные пространства
% (ОРП), в которых ранг любого элемента заведомо натурален.
\begin{frame}{Основные определения}
    $\{\Omega_i\}$ - естественное разбиаение на непересекающиеся классы, где
    
    $\Omega_1$ максимальные по включению элементы $\Omega$,
    
    $\Omega_2$ максимальные по включению элементы $\Omega {\setminus}{ \Omega_1}$ и тд.
    
    $rg(\Delta)=k$ при $\Delta\in\Omega_k$, где $k$ - ранг элемента $\Delta$.
\end{frame}

% 
% 
% 
\begin{frame}{Основные определения}
    \textbf{Определение 1.} Пространство $(X, \Omega, h)$ есть
    однородное размерностное пространство (ОРП),
    если   выполнены следующие условия:
    \begin{enumerate}
        \item 
            Условия $\Omega$-однородности, состоящие в том, 
            что:
            \begin{enumerate}
                \item  
                     Семейство $\Omega$ имеет вид 
                     $\Omega = \{\varnothing\} {\bigcup }\overset{\infty}{\underset{k=1}{\bigcup}} \Omega_k$ ,
                     где при каждом целом $k \geq 1$ семейство $\Omega_k$
                     состоит из непустых множеств,
                     образующих разбиение $X$.
                \item
                    Для каждого $E \in \Omega_k$ и каждого $l > k$
                    семейство 
                    $\Omega_l(E) = \{P \in \Omega_l : P \bigcap E \neq \varnothing\}$
                    образует  разбиение $E$ ,
                    причем количество элементов $| \Omega_l (E) |$ 
                    этого разбиения конечно и больше единицы.
                    Количество элементов множества $\Omega_1$
                    предпологается конечным (возможно равным единице)
                    или счетным.
                \item
                    Найдется константа $C_\Omega$ такая,
                    что при любых целых $k \geq 1$ и $l > k$ и любых 
                    $E_1, E_2 \in \Omega_k$
                    справедливо неравенство
                    $\frac{|\Omega_l(E_1)|}{|\Omega_l(E_2)|} < C_\Omega $.
            \end{enumerate}
    \end{enumerate}
\end{frame}

% 
% 
% 
\begin{frame}{Основные определения}
    \begin{enumerate}
    \setcounter{enumi}{1}
        \item 
            Условия $h$-однородности, состоящие в том, что:
            \begin{enumerate}
                \item 
                    $0 < h(E) \leq 1$ при любом непустом
                    $E \in \Omega $.
                \item
                    $\underset{k \to \infty} {\lim}
                    \sup \{ h(E) : E \in \Omega_k \} = 0$.
                \item
                    Существует константа $C_h$ такая,
                    что для любого целого $k \geq 1$ и для любых 
                    $E_1 , E_2 \in \Omega_k$,
                    справедливо неравенство 
                    $ \frac{h(E_1)}{h(E_2)} < C_h $.
            \end{enumerate}
    \end{enumerate}
\end{frame}

% 
% Поскольку семейства $\Omega_k$ однозначно восстанавливаются по 
% семейству $\Omega$, мы будем также использовать сокращенную запись
% 
\begin{frame}{Основные определения}
    $\mathfrak{A} = (X, \Omega, h)$ - фиксированное ОРП, где
    
    $\Omega = \{\varnothing\} {\bigcup }\overset{\infty}{\underset{k=1}{\bigcup}} \Omega_k$.
    Далее обозначим $\Omega = \{ \Omega_k \}_{k = 1}^\infty$.
    
    Пусть $0 < \alpha \in \mathbb{R}$.
    
    $\forall E \subset X, \forall n \in \mathbb{N}$
    рассмотрим $(X, \{ \Omega_k \}_{k = n}^\infty , h^\alpha)$, тогда
    
    $\mu_H (E) := \mu_n^\alpha (E) $
\end{frame}

% 
% 
% 
\begin{frame}{Основные определения}
    \textbf{Определение 2.} 
    $ \mu_{\mathfrak{A}}^\alpha (E) \mathrel{\stackrel{\rm def}=}
    \underset{n \to \infty}{\lim}
    \mu_n^\alpha (E)$ будем называть внешней $\alpha -$мерой Хаусдорфа произвольного множества $E \subset X$
    (допускается значения предела равное $\infty$).
    
    Размерностью Хаусдорфа множества $E \subset X$
    мы будем называть (возможно равную бесконечности) величину
    $
    \dim_H (E) =
    \dim_H(E, \mathfrak{A}) \mathrel{\stackrel{\rm def}=}
    \inf \{ \alpha > 0 : \mu_{\mathfrak{A}}^\alpha (E) = 0 \}
    $.
\end{frame}

% 
% 
% 
\begin{frame}{Основные определения}
    ${ \sigma}=\{ k_n \}_{n = 1}^\infty : k_i < k_{i+1}$ (далее
    обозначаем $\sigma \in \uparrow (\mathbb{N})$).
    \newline
    
    Соответственно положим ${\Omega_\sigma}=\{ \Omega_{k_n} \}_{n=1}^\infty$. Тогда  
    ${\mathfrak{A}_\sigma}=(X, \Omega_\sigma , h) - $  ОРП. 
    Для каждого $E \subset X$
    положим 
    $\mbox{Dim}_H (\mathfrak{A}, E) =
    \{ \dim_H (\mathfrak{A}_\sigma, E) : \sigma \in \uparrow
    (\mathbb{N}) \}$.
\end{frame}

% 
% 
% 
\begin{frame}{Основные определения}
    \textbf{Определение 3.} Множество $ \mbox{Dim}_H (\mathfrak{A}, E) $ -  спектром Хаусдорфа-Безиковича множества $E$ в пространстве $\mathfrak{A}$.
    
    В случае, если $ E=X $ множество
    $ \mbox{Dim}_H (\mathfrak{A}, E) $
    обозначим через $ \mbox{Dim}_H (\mathfrak{A})$.
    \newline
     
    Обозначим через $
    \mathcal{N}_l (E) =
    | \Omega_l (E) | =
    card \{ \Delta \in \Omega_l : \Delta \bigcap E \neq 
    {\varnothing}\} 
    $.
    
    Пусть
    $M_l = 
    \sup \{ h(\Delta) : \Delta \in \Omega_l \}$,
    $m_l =
    \inf \{ h(\Delta) : \Delta \in \Omega_l \}$.
    
    Тогда
    $
    \frac{M_l}{m_l} \leq C_h
    $, в силу определения ОРП.
\end{frame}

% Заметим, что, в силу однородности рассматриваемого пространства, 
% значение $d(\mathfrak{A}, E)$ при всех $E \in \Omega$
% одинаково; мы будем называть это значение верхней размерностью 
% пространства $\mathfrak{A}$.
% 
% Ясно, что в случае, если $\mathfrak{A}$ есть пространство
% $\mathbb{R}^n$, вместе с совокупностью  $\Omega$ всех 
% диадических кубов и длиной стороны куба, взятой в качестве
% значения $h$, то определенная выше размерность Минковского
% совпадает с обычной фрактальной размерностью множества $E$ в
% $\mathbb{R}^n$, а  величина $\dim_H (\mathfrak{A}, E)$ - с его
% размерностью Хаусдорфа.
%
% В общем случае справедливо неравенство $\dim_H (\mathfrak{A}, E)
% \leq d(\mathfrak{A}, E)$, а множество $\mbox{Dim}_H
% (\mathfrak{A},E)$,  как легко видеть, лежит между величинами
% $\dim_H (\mathfrak{A}, E)$ и $d(\mathfrak{A}, E)$. 
% В частности, если размерности Хаусдорфа и Минковского множества
% $E$ совпадают, то его спектр Хаусдофа-Безиковича состоит из одной
% точки. 
\begin{frame}{Основные определения}
    Положим 
    $d(E) =
    d(\mathfrak{A}, E) \mathrel{\stackrel{\rm def}=}
    \underset{l \to \infty}{\overline{\lim}}
    \frac{\ln{\mathcal{N}_l (E)}}{\ln{\frac{1}
    {m_l}}}
    $ - размерностью Минковского множества $E$ 
    в пространстве $\mathfrak{A}$.
    \newline
    
    $\dim_H (\mathfrak{A}, E) \leq
    \mbox{Dim}_H (\mathfrak{A},E) \leq
    d(\mathfrak{A}, E)$
\end{frame}

% Заметим, что конечный индекс компактности имеет, очевидно, пространство
% $\mathbb{R}^n$ с совокупностью  $\Omega$ всех диадических кубов. 
% Этот пример показывает, что требование убывания  последовательности
% $\{\Delta_i\}_{i=1}^\infty \subset \Omega$
% в пункте 2 определения 4 нельзя заменить  требованием убывания последовательности
% $\{\aleph(\Delta_i)\}_{i=1}^\infty $.
\begin{frame}{Основные определения}
    \textbf{Определение 4.} Пространство
    $(X,\Omega, h)$
    имеет конечный индекс компактности,
    если $\exists n \in \mathbb{N} : \forall \Delta \in \Omega
    \ \exists \aleph (\Delta)$
    являющееся объединением самого $\Delta$
    и не более, чем $n$ других элементов $\Omega$ того же ранга,
    что и $\Delta$,так, что выполнены следующие условия:
    \begin{enumerate}
        \item 
            Для любых $\Delta_1 \subset \Delta_2$
            из $\Omega$ выполнено
            $\aleph (\Delta_1) \subset \aleph(\Delta_2) $
        \item 
            Для любой убывающей последовательности
            $\{\Delta_i\}_{i=1}^\infty \subset \Omega$
            выполнено
            $\overset{\infty}{\underset{i = 1}{\bigcap}}
            \aleph (\Delta_i) \neq \varnothing$ 
        \item 
            Для каждого $\Delta \in \Omega$
            справедливо условие
            $
            \underset{l \to \infty}{\lim}
            \frac{ | \tilde{\Omega}_l(\Delta) | }
            { | \Omega_l(\Delta) | } =
            1
            $,
            где
            $
            \tilde{\Omega}_l(\Delta) =
            \{\Delta' \in  \Omega_l : 
            \aleph (\Delta') \subset \Delta \}
            $
    \end{enumerate}
\end{frame}

% 
% Сформулируем следующую основную теорему.
% 
\begin{frame}{Основные определения}
    \textbf{Теорема 2.} Пусть
    $\mathfrak{A} = (X, \Omega, h) - $ ОРП
    с конечным индексом компактности и верхней размерностью
    $d > 0$.
    \newline
    
    Пусть $J \subset [0, d] - $ компакт.
    \newline
    
    Тогда существуют $\sigma  \in \uparrow (\mathbb{N}) $
    и $E \subset X$ такие, что $\mbox{Dim}_H (\mathfrak{A_\sigma}, E) = J$.
\end{frame}


%%%%%%%%%%%%%%%%%%%%%%%%%%%%%%%%%%%%%%%%%%%%%%%%%%%%%%%%%%%%%%%%%%%%%%%%%%%%%%%%%%%%%%%%%%%%%%%%%%
%% Вспомогательные построения %%
%%%%%%%%%%%%%%%%%%%%%%%%%%%%%%%%%%%%%%%%%%%%%%%%%%%%%%%%%%%%%%%%%%%%%%%%%%%%%%%%%%%%%%%%%%%%%%%%%%

% 
% 
% 
\begin{frame}{Вспомогательные построения}
    \textbf{Лемма 1.} Пусть выполнены условия теоремы 2, 
    $k \in \mathbb{N}$, $\mathfrak{E} \subset \Omega_k$,
    $card (\mathfrak{E}) < \infty$. Пусть
    $\beta < \alpha  $ - два числа из
    $(0, d(\mathfrak{A})).$
    Тогда существуют натуральное число $l > k$, натуральное число $\mathcal{N}$
    и множество $G = G(\alpha, \beta, \mathfrak{E})$, лежащее в $\Omega_l$,
    такие, что
    \begin{enumerate}
        \item 
            Для каждого $\Delta ' \in G$ существует
            $\Delta \in \mathfrak{E}$ такое, что 
            $\Delta' \in \Omega_l (\Delta)$.
        \item 
            Для каждого $\Delta \in \mathfrak{E}$
            выполнено:
            $card( G(\alpha, \beta, \mathfrak{E})
            \bigcap \Omega_l (\Delta)) = \mathcal{N}$.
        \item 
            Для каждого $\Delta' \in 
            G(\alpha, \beta, \mathfrak{E}) $
            найдется $\Delta \in \mathfrak{E}$ такое,
            что $\aleph (\Delta')
            \subset \Delta$.
        \item 
            Для каждого $\Delta \in \mathfrak{E}$
            справедливо неравенство
            $\underset{\Delta' \in
            G(\alpha, \beta, \mathfrak{E}) \bigcap
            \Omega_l (\Delta)}{\sum}
            h^\beta(\Delta')   > 1$.
        \item 
            Справедливо неравенство $\underset{\Delta' \in
            G}{\sum}
            h^\alpha(\Delta')   < 1$.
    \end{enumerate}
\end{frame}

% 
% 
% 
\begin{frame}{Вспомогательные построения}
    \textbf{Замечание к лемме 1.} Множество 
    $G(\alpha, \beta, \mathfrak{E})$
    определенное в Лемме 1 неоднозначно
    (далее обозначаем это символом 
    $G \in \mathfrak{J} (\alpha, \beta, \mathfrak{E})$).
    \newline
    
    Заметим, в частности, что если
    $G \in \mathfrak{J} (\alpha, \beta, \mathfrak{E})$ и
    $G_1 \in \mathfrak{J} (\alpha_1, \beta_1, G)$
    то
    $G_1 \in \mathfrak{J} (\alpha_1, \beta_1, \mathfrak{E})$.
\end{frame}

% (такие $\mathfrak{E}_n$ существуют, разумеется, не для всех $\sigma$).
% 
% Утверждение леммы остается в силе, если вместо равенства 
% $\mathfrak{E}_n =
% G(\alpha_n, \beta_n, \mathfrak{E}_{n - 1})$
% выполнено условие
% $\mathfrak{E}_n \in \mathfrak{J} (\alpha_n, \beta_n, \mathfrak{E}_{n-1})$.
\begin{frame}{Вспомогательные построения}
    \textbf{Лемма 2.} Пусть $\mathfrak{A}$ -ОРП с конечным
    индексом компактности,
    $\sigma  \in \uparrow (\mathbb{N}), \ \alpha_n > \beta_n > 0 - $
    последовательности чисел из промежутка $(0, d(\mathfrak{A}) ) : 
    \beta_n - \alpha_n \to 0$.
    
    Пусть $k_0 = 1, \mathfrak{E}_{0} - $
    произвольное конечное подсемейство в $\Omega_1, 
    \mathfrak{E}_{n} - $
    конечные подсемейства в $\Omega_{k_n}$,
    причем $\mathfrak{E}_n = 
    G(\alpha_n, \beta_n, \mathfrak{E}_{n-1}), n \geq 1$.
    
    Пусть $E_n - $ объединение всех множеств,
    входящих в $\mathfrak{E}_n$,
    
    $ E = 
    \overset{\infty}{\underset{i=1}{\bigcap}}
    E_n$.
    \newline
    
    Тогда 
    $\dim_H (\mathfrak{A}_\sigma, E)=
    \underset{n \to \infty}{\underline{\lim}}\alpha_n$
\end{frame}

%%%%%%%%%%%%%%%%%%%%%%%%%%%%%%%%%%%%%%%%%%%%%%%%%%%%%%%%%%%%%%%%%%%%%%%%%%%%%%%%%%%%%%%%%%%%%%%%%%
%% Доказательство теоремы 2 %%
%%%%%%%%%%%%%%%%%%%%%%%%%%%%%%%%%%%%%%%%%%%%%%%%%%%%%%%%%%%%%%%%%%%%%%%%%%%%%%%%%%%%%%%%%%%%%%%%%%
\begin{frame}{Доказательство теоремы 2}
    $\mathfrak{E}_0 \subset \Omega_1$ - произвольное конечно;
    $\alpha_n \in (0, d( \mathfrak{A} )) $, 
    с множесвтом частичных пределов $J$;
    $\beta_n = \alpha_n - \frac{\alpha_n}{n}$.
    
    По лемме 1 по индукции найдем 
    $k_n \in \  \uparrow(\mathbb{N})$ и
    $\mathfrak{E}_n \subset \Omega_{k_n} : \mathfrak{E}_n =
    G(\alpha_n, \beta_n, \mathfrak{E}_{n-1}) \ \forall n \in \mathbb{N}$.
    
    Тогда, для любой подпоследовательности
    $\{ k_{n_j} \}_{j = 1}^\infty$ имеем
    $\mathfrak{E}_{n_j} \in
    \mathfrak{J} (\alpha_{n_j}, \beta_{n_j}, \mathfrak{E}_{n_{j - 1}})$,
    а значит по лемме 2
    $\dim_H(  \mathfrak{A}_{\{ k_{n_j} \}},
    \overset{\infty}{
        \underset{j = 1}{
            \bigcap
        }
    }
    E_{n_j}
    ) =
    \underset{j \to \infty}{\underline{\lim}} \ \alpha_{n_j}
    $.
    
    $\forall n_j$ выполнено
    $
    \overset{\infty}{
        \underset{j = 1}{
            \bigcap
        }
    }
    E_{n_j} = E
    $;
    множество же нижних пределов всех
    подпоследовательностей последовательности
    совпадает с $J$.

    Итак
    $\mbox{Dim}_H 
    (\mathfrak{A}_{\{ k_{n} \}} , E) =
    J$.
    
    Теорема доказана.
\end{frame}

% 
% 
% 
\begin{frame}{Доказательство теоремы 2}
    $\mathfrak{E}_0 \subset \Omega_1$ - произвольное конечно;
    $\alpha_n \in (0, d( \mathfrak{A} )) $, 
    с множесвтом частичных пределов $J$;
    $\beta_n = \alpha_n - \frac{\alpha_n}{n}$.
    
    По лемме 1 по индукции найдем 
    $k_n \in \  \uparrow(\mathbb{N})$ и
    $\mathfrak{E}_n \subset \Omega_{k_n} : \mathfrak{E}_n =
    G(\alpha_n, \beta_n, \mathfrak{E}_{n-1}) \ \forall n \in \mathbb{N}$.
    
    Тогда, для любой подпоследовательности
    $\{ k_{n_j} \}_{j = 1}^\infty$ имеем
    $\mathfrak{E}_{n_j} \in
    \mathfrak{J} (\alpha_{n_j}, \beta_{n_j}, \mathfrak{E}_{n_{j - 1}})$,
    а значит по лемме 2
    $\dim_H(  \mathfrak{A}_{\{ k_{n_j} \}},
    \overset{\infty}{
        \underset{j = 1}{
            \bigcap
        }
    }
    E_{n_j}
    ) =
    \underset{j \to \infty}{\underline{\lim}} \ \alpha_{n_j}
    $.
    
    $\forall n_j$ выполнено
    $
    \overset{\infty}{
        \underset{j = 1}{
            \bigcap
        }
    }
    E_{n_j} = E
    $;
    множество же нижних пределов всех
    подпоследовательностей последовательности
    совпадает с $J$.

    Итак
    $\mbox{Dim}_H 
    (\mathfrak{A}_{\{ k_{n} \}} , E) =
    J$.
    
    Теорема доказана.
\end{frame}

%%%%%%%%%%%%%%%%%%%%%%%%%%%%%%%%%%%%%%%%%%%%%%%%%%%%%%%%%%%%%%%%%%%%%%%%%%%%%%%%%%%%%%%%%%%%%%%%%%
%% Замечание к теореме 2 %%
%%%%%%%%%%%%%%%%%%%%%%%%%%%%%%%%%%%%%%%%%%%%%%%%%%%%%%%%%%%%%%%%%%%%%%%%%%%%%%%%%%%%%%%%%%%%%%%%%%
\begin{frame}{Теорема 2}
    \textbf{Замечание к теореме 2.} Построенное в теореме 2  $(E,
    \{
    \Delta \bigcap E:
    \Delta \in \mathfrak{E}_n
    \}_{n = 1}^\infty 
    ,
    h_E)
    $,
    где $h_E (\Delta \bigcap E)  \mathrel{\stackrel{\rm def}=}
    h(\Delta)$,
    является ОРП,
    которое можно, тем самым, рассматривать как
    подпространство исходного пространства $\mathfrak{A}$.
\end{frame}

%%%%%%%%%%%%%%%%%%%%%%%%%%%%%%%%%%%%%%%%%%%%%%%%%%%%%%%%%%%%%%%%%%%%%%%%%%%%%%%%%%%%%%%%%%%%%%%%%%
%% Замечание к теореме 2 %%
%%%%%%%%%%%%%%%%%%%%%%%%%%%%%%%%%%%%%%%%%%%%%%%%%%%%%%%%%%%%%%%%%%%%%%%%%%%%%%%%%%%%%%%%%%%%%%%%%%
\begin{frame}{Теорема 2}
    Теорема 2 означает, что спектр Безиковича-Хаусдорфа у
    подпространства произвольного фиксированного ОРП $\mathfrak{A}$
    с конечным индексом компактности
    может быть проивольным замкнутым множеством,
    лежащим в $[0, d(\mathfrak{A})]$.
\end{frame}
%%%%%%%%%%%%%%%%%%%%%%%%%%%%%%%%%%%%%%%%%%%%%%%%%%%%%%%%%%%%%%%%%%%%%%%%%%%%%%%%%%%%%%%%%%%%%%%%%%
%%%%%%%%%%%%%%%%%%%%%%%%%%%%%%%%%%%%%%%%%%%%%%%%%%%%%%%%%%%%%%%%%%%%%%%%%%%%%%%%%%%%%%%%%%%%%%%%%%
%%%%%%%%%%%%%%%%%%%%%%%%%%%%%%%%%%%%%%%%%%%%%%%%%%%%%%%%%%%%%%%%%%%%%%%%%%%%%%%%%%%%%%%%%%%%%%%%%%

\section{Литература}

\begin{frame}{Литература}
    \begin{enumerate}
    	\item 
    	    Г.А. Леонов, А.А.  Флоринский. Об оценках обобщенной размерности Хаусдорфа. MSC 28C99, УДК 517.987, Вестник СПбГУ. Сер. 1, Т. , 2019 , вып. 3
    \end{enumerate}
\end{frame}


%%%%%%%%%%%%%%%%%%%%%%%%%%%%%%%%%%%%%%%%%%%%%%%%%%%%%%%%%%%%%%%%%%%%%%%%%%%%%%%%%%%%%%%%%%%%%%%%%%
%%%%%%%%%%%%%%%%%%%%%%%%%%%%%%%%%%%%%%%%%%%%%%%%%%%%%%%%%%%%%%%%%%%%%%%%%%%%%%%%%%%%%%%%%%%%%%%%%%
%%%%%%%%%%%%%%%%%%%%%%%%%%%%%%%%%%%%%%%%%%%%%%%%%%%%%%%%%%%%%%%%%%%%%%%%%%%%%%%%%%%%%%%%%%%%%%%%%%

%% ДОП СЛАЙДЫ %%

% Внешняя мера
\begin{frame}{Дополнительные утверждения}
    Пусть $X$ - фиксированное множество.
    Внешней мерой называется функция $f : 2^X \to [0, \infty]$ такая, что
    \begin{enumerate}
        \item[$\bullet$] $f(\varnothing) = 0$
        \item[$\bullet$] $\forall A \subseteq X$,
        $\forall A_n \subset X, n \geq 1,
        A \subseteq \overset{\infty}{ \underset{n = 1}{ \bigcup } } A_n :
        f(A) \leq \overset{\infty}{ \underset{n = 1}{ \sum } } f(A_n)$
    \end{enumerate}
\end{frame}

\end{document}

